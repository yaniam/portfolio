\documentclass[a4paper,10pt]{article}

\usepackage[margin=1in]{geometry}
\usepackage{helvet}
\renewcommand{\familydefault}{\sfdefault}
\usepackage{fontawesome5}
\usepackage[hidelinks]{hyperref}
\usepackage{titlesec}
\usepackage{enumitem}
\usepackage{setspace}
\setstretch{1.1}
\pagenumbering{gobble}

% ---------- Section formatting ----------
\titleformat{\section}{\large\bfseries\uppercase}{\thesection}{0.5em}{}
\titlespacing{\section}{0pt}{8pt}{4pt}
\newcommand{\sectionrule}{\vspace{-3pt}\hrule\vspace{3pt}}

% ---------- Header ----------
\begin{document}

\begin{center}
    {\LARGE \textbf{IAN PORTO E MELLO}}\\[2pt]
    Brasília, Federal District, Brazil\\[3pt]
    \faEnvelope\ \href{mailto:ianporto25@gmail.com}{ianporto25@gmail.com} \quad
    \faLinkedin\ \href{https://www.linkedin.com/in/ian-porto-mello}{linkedin.com/in/ian-porto-mello} \quad
    \faGithub\ \href{https://github.com/yaniam}{GitHub} \quad
    \faGlobe\ Portfolio
\end{center}

\vspace{0.8em}
\hrule
\vspace{0.8em}

% ---------- Objective ----------
\section*{Objective}
\sectionrule
To work as a \textbf{Machine Learning Engineer}, \textbf{Computer Vision Engineer} or \textbf{Data Engineer}, applying advanced AI, data engineering, and data processing techniques to develop robust, efficient, and interpretable systems for real-world applications.

% ---------- Professional Summary ----------
\section*{Professional Summary}
\sectionrule
AI and Data Engineer with strong experience in \textbf{Machine Learning, Computer Vision, and Data Engineering}, currently pursuing an M.Sc. in Computer Science at the \textbf{University of Brasília (UnB) focused on Point Cloud and Mesh Quality Assessment}.  
Experienced in the full lifecycle of AI systems—from data collection and preprocessing to model deployment and monitoring.  
Proficient in \textbf{PyTorch, TensorFlow, Pandas, OpenCV, Open3D, Python, SQL}. 

Combines technical rigor with a product-driven mindset, optimizing for scalability, latency, and cost.

% ---------- Experience ----------
\section*{Professional Experience}
\sectionrule

\textbf{Banco do Brasil} \hfill \textit{Brasília, DF — Mar 2024 – Present}\\
\textit{IT Systems Analyst}
\begin{itemize}[leftmargin=1.2em, itemsep=1pt]
    \item Oversee nationwide IT operations, ensuring real-time visibility into the performance and availability of equipment across hundreds of banking branches.
    \item Design and maintain internal \textbf{data pipelines} and \textbf{relational databases} (PostgreSQL, SQL Server) to track hardware lifecycles, configuration states, and incident logs.
    \item Develop \textbf{dashboards and data science models} for predictive maintenance, capacity planning, and crisis detection using Python, Pandas, and visualization tools.
    \item Integrate heterogeneous data sources to support decision-making and automate field monitoring tasks, improving response time and operational reliability.
    \item Coordinate with infrastructure and analytics teams to identify anomalies, mitigate failures, and ensure compliance with corporate SLAs.
    \item Participate in \textbf{incident response and crisis management}, using data-driven insights to prioritize repairs, forecast impact, and maintain business continuity during outages.
    \item Lead initiatives in process automation, reporting standardization, and data quality governance across distributed environments.

\end{itemize}

\vspace{0.4em}
\textbf{CENSIPAM (Sistema de Proteção da Amazônia)} \hfill \textit{Brasília, DF — Sep 2021 – Dec 2023}\\
\textit{Machine Learning Engineer}
\begin{itemize}[leftmargin=1.2em, itemsep=1pt]
    \item Developed and maintained \textbf{machine learning systems} for autonomous environmental monitoring of the Amazon region.
    \item Designed scalable \textbf{data ingestion and preprocessing pipelines} for satellite, meteorological, and IoT sensor data.
    \item Trained and validated \textbf{temporal classification models} for early detection of forest fires under varying environmental conditions.
    \item Integrated model outputs into operational dashboards to support decision-making, emergency alerts, and sustainability analysis.
    \item Collaborated with multidisciplinary teams to ensure the reliability, interpretability, and maintainability of deployed AI solutions.
\end{itemize}


\vspace{0.4em}
\textbf{UnB Internet of Things (UIoT) Research Group} \hfill \textit{Brasília, DF — Apr 2021 – Jan 2023}\\
\textit{Undergraduate Researcher – Computer Vision \& Embedded AI}
\begin{itemize}[leftmargin=1.2em, itemsep=1pt]
    \item Conducted research on \textbf{computer vision systems} for crowd detection and behavior monitoring using real-time video streams.
    \item Designed and implemented \textbf{object detection and pattern recognition pipelines} with Python, OpenCV, and TensorFlow.
    \item Built \textbf{embedded AI architectures} operating on IoT and Fog Computing environments for edge-based image analysis and event classification.
    \item Developed algorithms for \textbf{social distancing and density estimation}, integrating image processing with sensor fusion data.
    \item Co-authored academic publications on \textbf{visual analytics, edge intelligence, and collaborative perception systems for IoT networks.}

\end{itemize}

% ---------- Education ----------
\section*{Education}
\sectionrule

\textbf{University of Brasília (UnB)} \hfill \textit{Feb 2025 – Feb 2027}\\
\textit{M.Sc. in Computer Science}\\
Research in Point Cloud and Mesh Quality Assessment (PCQA/MQA).

\vspace{0.4em}
\textbf{University of Brasília (UnB)} \hfill \textit{Jan 2017 – Dec 2023}\\
\textit{B.Sc. in Mechatronics, Control and Automation Engineering}

% ---------- Skills ----------
\section*{Technical Skills}
\sectionrule
\begin{itemize}[leftmargin=1.2em, itemsep=1pt]
    \item \textbf{AI/ML Frameworks:} PyTorch, TensorFlow/Keras, Scikit-Learn
    \item \textbf{Data Engineering:} Pandas, NumPy, SQL, PySpark, Airflow
    \item \textbf{3D \& Computer Vision:} OpenCV, Open3D, Point Cloud/Mesh
    \item \textbf{Programming:} Python, C, C++, Rust
    \item \textbf{Soft Skills:} Communication, Interdisciplinary collaboration, Adaptability, Product mindset
\end{itemize}

\section*{Languages}
\sectionrule
\begin{itemize}[leftmargin=1.2em, itemsep=1pt]
    \item \textbf{Portuguese:} Native
    \item \textbf{English:} Fluent (Cambridge C1 Advanced or above – CEFR)
\end{itemize}
% ---------- Publications ----------
\section*{Publications}
\sectionrule
\begin{itemize}[leftmargin=1.2em, itemsep=3pt]

    \item \textit{Sistema de aprendizado de máquina para tipificação de incêndios florestais na Amazônia com série temporal}\\
    (Machine Learning System for the Typification of Amazon Forest Fires Using Time Series Data)

    \item \textit{Sistema para a identificação de aglomerações operando em Redes IoT e Fog Computing}\\
    (System for the Identification of Crowds Operating on IoT and Fog Computing Networks)

    \item \textit{Sistema monitor de aglomerações baseado em reconhecimento de padrões e cálculos de distanciamento social operante em rede IoT estruturada em Fog Computing}\\
    (Crowd Monitoring System Based on Pattern Recognition and Social Distance Estimation in IoT Networks Structured with Fog Computing)
\end{itemize}

\end{document}
